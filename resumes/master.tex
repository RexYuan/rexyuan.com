%%%%%%%%%%%%%%%%%%%%%%%%%%%%%%%%%%%%%%%%%
% Medium Length Professional CV
% LaTeX Template
% Version 2.0 (8/5/13)
%
% This template has been downloaded from:
% http://www.LaTeXTemplates.com
%
% Original author:
% Trey Hunner (http://www.treyhunner.com/)
%
% Important note:
% This template requires the resume.cls file to be in the same directory as the
% .tex file. The resume.cls file provides the resume style used for structuring the
% document.
%
%%%%%%%%%%%%%%%%%%%%%%%%%%%%%%%%%%%%%%%%%

%----------------------------------------------------------------------------------------
%	PACKAGES AND OTHER DOCUMENT CONFIGURATIONS
%----------------------------------------------------------------------------------------

\documentclass{resume} % Use the custom resume.cls style

\usepackage[left=0.5in,top=0.5in,right=0.5in,bottom=0.5in]{geometry} % Document margins

\name{Chih-cheng Rex Yuan} % Your name
%\address{123 Broadway \\ City, State 12345} % Your address
%\address{123 Pleasant Lane \\ City, State 12345} % Your secondary addess (optional)
\address{hello@rexyuan.com} % Your phone number and email

\begin{document}

%----------------------------------------------------------------------------------------
%	EDUCATION SECTION
%----------------------------------------------------------------------------------------

\begin{rSection}{Publication}

{\bf Ensuring Fairness with Transparent Auditing of Quantitative Bias in AI Systems} \hfill {2024} \\
{\em (to appear) IEEE 2024 Pacific Neighborhood Consortium Annual Conference and Joint Meetings (PNC)}

\end{rSection}

\begin{rSection}{Education}

{\bf National Taiwan Normal University} \hfill {2014 - 2019} \\
{\em B.S. in Computer Science and Information Engineering} \\
{\bf FLOLAC Summer School at National Taiwan University} \hfill {2017, 2018, 2019} \\
{\bf CONFESTA Summer School at University of Chinese Academy of Sciences} \hfill {2018}

\end{rSection}

%----------------------------------------------------------------------------------------
%	WORK EXPERIENCE SECTION
%----------------------------------------------------------------------------------------

\begin{rSection}{Experience}

\begin{rSubsection}{Institute of Information Science, Academia Sinica}{Taipei}{Research Assistant}{2023 - }
\item Generative AI fairness research: program synthesis experiments with ChatGPT and Frama-C.
\item AI fairness research: development of an auditing framework and analysis of COMPAS dataset using the framework.
\end{rSubsection}

\begin{rSubsection}{The News Lens}{Taipei}{Software Engineer}{2023}
\item Website development in JavaScript with Next.js.
\item LLM integration with LangChain.
\end{rSubsection}

\begin{rSubsection}{SiFive}{Hsinchu}{Hardware Engineer}{2022 - 2023}
\item Maintenance of assertion-based verification monitors of the cache communication protocol TileLink2 in Chisel of Scala.
\item Development of regression automation tools for system validation in Bash and Perl.
\end{rSubsection}

\begin{rSubsection}{Formal Land}{Paris}{Software Engineer}{2022}
\item Formal verification of the OCaml codebase of Tezos economic protocol Jakarta in Coq.
\end{rSubsection}

\begin{rSubsection}{Institute of Information Science, Academia Sinica}{Taipei}{Research Assistant}{2019 - 2022}
\item Developed bit-level hardware verification safety algorithm based on algorithmic boolean learning and
its implementation which solves about 300 of 747 instances in HWMCC10 benchmark.
Optimized implementation with improved algorithm which combines learning and interpolation method solves about 400 instances.
\item AIG input parser, solver interface, program data structures, profiling and logging,
memory arena, and backend algorithm engines are independently implemented in C++ with minisat.
\end{rSubsection}

\begin{rSubsection}{Institute of Information Science, Academia Sinica}{Taipei}{Adjunct Research Assistant}{2017 - 2019}
\item SAT-solving non-trivial reduction of Poset Cover Problem for Message Sequence Chart synthesis implemented in Python with z3.
\item NP-hardness of k-anonymity Checking Problem and a randomized algorithm thereof for the 2010 Taiwanese Population And Housing Census data implemented in Python with MySQL.
\end{rSubsection}

\end{rSection}

%----------------------------------------------------------------------------------------
%	TECHNICAL STRENGTHS SECTION
%----------------------------------------------------------------------------------------

%----------------------------------------------------------------------------------------
%	EXAMPLE SECTION
%----------------------------------------------------------------------------------------

%\begin{rSection}{Section Name}

%Section content\ldots

%\end{rSection}

%----------------------------------------------------------------------------------------

\end{document}
