%%%%%%%%%%%%%%%%%%%%%%%%%%%%%%%%%%%%%%%%%
% Medium Length Professional CV
% LaTeX Template
% Version 2.0 (8/5/13)
%
% This template has been downloaded from:
% http://www.LaTeXTemplates.com
%
% Original author:
% Trey Hunner (http://www.treyhunner.com/)
%
% Important note:
% This template requires the resume.cls file to be in the same directory as the
% .tex file. The resume.cls file provides the resume style used for structuring the
% document.
%
%%%%%%%%%%%%%%%%%%%%%%%%%%%%%%%%%%%%%%%%%

%----------------------------------------------------------------------------------------
%	PACKAGES AND OTHER DOCUMENT CONFIGURATIONS
%----------------------------------------------------------------------------------------

\documentclass{resume} % Use the custom resume.cls style

\usepackage[left=0.5in,top=0.5in,right=0.5in,bottom=0.5in]{geometry} % Document margins

\name{Chih-cheng Yuan} % Your name
%\address{123 Broadway \\ City, State 12345} % Your address
%\address{123 Pleasant Lane \\ City, State 12345} % Your secondary addess (optional)
\address{chihchengyuan@gmail.com} % Your phone number and email

\begin{document}

%----------------------------------------------------------------------------------------
%	EDUCATION SECTION
%----------------------------------------------------------------------------------------

\begin{rSection}{Education}

{\bf National Taiwan Normal University} \hfill {2014 - 2019} \\
{\em B.S. in Computer Science and Information Engineering}

{\bf FLOLAC Summer School at National Taiwan University} \hfill {2017, 2018, 2019} \\
{\bf CONFESTA Summer School at University of Chinese Academy of Sciences} \hfill {2018}

\end{rSection}

%----------------------------------------------------------------------------------------
%	WORK EXPERIENCE SECTION
%----------------------------------------------------------------------------------------

\begin{rSection}{Experience}

\begin{rSubsection}{Institute of Information Science, Academia Sinica}{2019 - Present}{Research Assistant}{}
\item (On-going) Developed bit-level hardware verification safety algorithm based on algorithmic boolean learning and
its implementation which solves about 300 of 747 instances in HWMCC10 benchmark.
Optimized implementation with improved algorithm which combines learning and interpolation method solves about 400 instances,
where Berkeley's state-of-the-art tool ABC's interpolation algorithm solves around 600 instances on the same computer.

AIG input parser, solver interface, program data structures, profiling and logging, and backend algorithm engines
are independently implemented in C++ with minisat.
\end{rSubsection}

%------------------------------------------------

\begin{rSubsection}{Institute of Information Science, Academia Sinica}{2017 – 2019}{Adjunct Research Assistant}{}
\item SAT-solving reduction of poset cover problem for message sequence chart synthesis  implemented in Python with z3.
\item NP-hardness of k-anonymity checking problem and a randomized approach thereof for the 2010 Taiwanese census data
implemented in Python with MySQL.
\end{rSubsection}

%------------------------------------------------

\begin{rSubsection}{Inside.com.tw}{2016}{Software Engineering Intern}{}
\item Backend Wordpress theme development; collaboration  and communication with frontend team.
\item In-house editor user-interface feature requests for editorial team.
\end{rSubsection}

%------------------------------------------------

\begin{rSubsection}{Inside.com.tw}{2015}{Editorial Intern}{}
\item Coverage, interviewing, and reporting on technology and startup related scenes.
\end{rSubsection}

\end{rSection}

%----------------------------------------------------------------------------------------
%	TECHNICAL STRENGTHS SECTION
%----------------------------------------------------------------------------------------

\begin{rSection}{Awards \& Certificates}

{\bf Academic Excellence Award} \hfill {2016} \\
{\em Ranked top 5\% in a class} \\
{\bf TOEFL iBT} \hfill {2016} \\
{\em Score: 108 (Reading: 29 / Listening: 30 / Speaking: 22 / Writing: 27)} \\
{\bf TOEIC} \hfill {2013} \\
{\em Score: 960}

\end{rSection}

%----------------------------------------------------------------------------------------
%	EXAMPLE SECTION
%----------------------------------------------------------------------------------------

%\begin{rSection}{Section Name}

%Section content\ldots

%\end{rSection}

%----------------------------------------------------------------------------------------

\end{document}
